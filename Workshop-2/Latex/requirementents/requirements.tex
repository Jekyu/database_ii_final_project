\section{Requirements Documentation}
\subsection{Functional Requirements}
\begin{itemize}
    \item FR1. Allow users to register with their details and relevant information (email address, personal details, and other details to be agreed upon).
    \item FR2. Allow already registered users to log in, taking into account their log-in credentials (email and password).
    \item FR3. Allow users to manage their storage space using folders, which must have specific options (view, organize, delete, move). Similarly, it must be possible to navigate between folders by hierarchical level according to the user's organization, listing the file elements of various permitted extensions, described specifically in ‘RF4’, which are classified in each of the folder layers.
    \item FR4. FR4. Users must be able to upload files to their workspace, organizing and separating items as they wish into folders. Possible events during file upload must be guaranteed, indicating whether the files have been uploaded correctly or whether errors have occurred during upload. Valid file types for user upload are (.docx, .pdf, .xls, .mp4, .mp3, .png, .jpg, .gif, .pptx) and the maximum upload size per file is 20 MB.
    \item FR5. Users must be able to view the current capacity of their storage space, detailing how much space is currently occupied and how much space remains available for use. It should be noted that this storage space depends directly on the user's plan.
    \item FR6. Allow the user to view important data about the files they have in their storage space, including file name, size, root or file type, and upload date. In addition, users should also be allowed to download files available in their storage space.
    However, Allow users to manage their files (delete, move, hide, and organize).
    \item FR7. Users shall have access only to files belonging to them and shall not have access to or be able to view items from other existing accounts.
    \item FR8. Users must have the ability to securely recover passwords (email with temporary token). This allows users to reset their passwords securely and reliably.
    \item FR9. Users may access a premium plan by making a fixed payment to purchase membership, which will have a predetermined value and will be valid for only six months. Users may have only one active premium membership, which allows them to expand the storage available on normal free user accounts
\end{itemize}
\subsection{Non-Functional Requirements}
Non-functional requirements establish the quality attributes and operational constraints of the file storage platform, each requirement has been refined to ensure technical feasibility and measurable implementation within the project scope.

\begin{itemize}
    \item \textbf{NFR1. Security.} 
    The system must ensure user authentication and data protection. Passwords must be stored securely using hash-based encryption, e.g., SHA-256, and all data transfers like login, upload, download must occur over HTTPS. Access control mechanisms will restrict unauthorized access to user content.

    \item \textbf{NFR2. Scalability.}
    The platform should be designed to support a moderate increase in users and files without significant performance loss. The implementation will rely on cloud storage services e.g., AWS S3 that allow horizontal scaling when necessary.

    \item \textbf{NFR3. Availability.}
    The prototype must maintain stable operation during normal usage and testing sessions. Full 24/7 availability is not required at this stage, but the system should be resilient to basic interruptions and restart gracefully.

    \item \textbf{NFR4. Performance.}
    Core functions such as login, file upload, and download should respond within acceptable limits, thinkink at least in seconds for standard files. Database queries and file retrieval will be optimized for concurrent user requests.

    \item \textbf{NFR5. Usability.}
    The user interface must be simple, intuitive, and accessible on desktop and mobile browsers. Navigation and file management tasks should be easily performed without user training.

    \item \textbf{NFR6. Maintainability.}
    The system architecture must follow a modular design, allowing future updates and debugging with minimal effort. Code documentation and naming conventions will ensure ease of maintenance.

    \item \textbf{NFR7. Compatibility.}
    The web application must operate correctly across major browsers like Chrome, Firefox or Edge. Responsive design must ensure usability on mobile devices.

    \item \textbf{NFR8. Reliability.}
    The system should handle minor failures without losing user data. Uploads and downloads must include confirmation and error-handling mechanisms. Advanced redundancy is beyond the current scope but will be considered for future iterations.

    \item \textbf{NFR9. Backup and Recovery.}
    Basic backup mechanisms must be implemented using cloud storage versioning or scheduled export of metadata. Recovery from file deletion or corruption should be possible during testing.

    \item \textbf{NFR10. Logging and Auditability.}
    The platform must record relevant user operations, login, upload, download and delete in an event log to support monitoring and debugging. Logs will be stored securely and reviewed during system evaluation.

    \item \textbf{NFR11. Interoperability.}
    The platform must provide RESTful APIs that enable future integration with external applications, such as institutional systems or educational platforms. Initial endpoints will focus on authentication and file access.
\end{itemize}
