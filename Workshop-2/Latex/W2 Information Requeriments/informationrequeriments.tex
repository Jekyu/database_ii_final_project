\section{Information Requirements}

This section identifies and describes the main types of information for the file storage platform that must retrieve and manage to support its business processes and user interactions.

\subsection{Main Information Types}

\begin{itemize}
    \item \textbf{User Account Information.} 
    Includes essential user details such as email, password, registration date, and status. This information supports authentication and user management processes. 
    It relates to user stories \textit{UH-01 User Registration}, \textit{UH-02 User Login}, and \textit{UH-03 Secure Password Recovery}.

    \item \textbf{Personal Profile Data.} 
    Contains user-identifying information. Name, last name, birthdate, contact data. 
    This data improves personalization and communication features, aligning with the \textit{Customer Relationship} block in the Business Model.

    \item \textbf{File Metadata.} 
    Describes the properties of each stored file: name, size, format, upload date, and folder location. 
    It enables browsing, searching, and organizing functionalities. 
    Directly related to user stories \textit{UH-04 Upload a File}, \textit{UH-05 Create and Navigate Folders}, and \textit{UH-06 View File List and Details}.

    \item \textbf{Storage Usage Data.} 
    Includes information about total and available space per user and plan type. 
    This supports the system’s scalability control and provides feedback to users about their consumption levels, as seen in \textit{UH-09 View Storage Capacity}.

    \item \textbf{Access Control and Permissions.} 
    Defines which user has access to each file or folder for view, edit or share. 
    Ensures security and privacy according to user roles and authentication tokens. 
    This type of data supports \textit{UH-07 Download a File}, \textit{UH-08 Delete a File or Folder}, and \textit{UH-11 Session Control and Security}.

    \item \textbf{Subscription and Payment Records.} 
    Stores plan details, payment methods, and subscription validity periods. 
    It enables the implementation of the freemium model defined in the \textit{Revenue Streams} block and supports future billing automation. 
    It relates to user story \textit{UH-09 View Storage Capacity} and the business requirement of premium upgrades.

    \item \textbf{Activity Logs.} 
    Logs all relevant system events, including uploads, downloads, deletions, and failed access attempts. 
    This information supports auditing, reliability, and fault diagnosis, as defined in non-functional requirements \textit{NFR10 Logging and Auditability} and user story \textit{UH-12 File Activity History}.

    \item \textbf{API Integration Data.} 
    Contains the tokens, permissions, and endpoints associated with third-party system access. 
    It supports the integration described in the \textit{Interoperability} requirement and relates to user story \textit{UH-13 API for Third-Party Integration}.
\end{itemize}

Each type of information is strategically connected to the business model and user experience:

\begin{itemize}
    \item \textbf{Customer Segments and Relationships:} 
    User and profile information support personalized and secure access for different user categories, for an initial case: students, professionals, and small businesses.
    \item \textbf{Value Proposition:} 
    File metadata, ACLs, and activity logs guarantee reliable and organized file management, reinforcing the core service value.
    \item \textbf{Revenue Streams:} 
    Subscription and payment data ensure the sustainability of the freemium and premium plans.
    \item \textbf{Key Activities:} 
    Monitoring user behavior and storage usage enables continuous system optimization and scalability planning.
\end{itemize}

Whith these information requirements that define the data backbone of the system, guiding the subsequent database query design and ensuring consistency between storage, performance, and business objectives.